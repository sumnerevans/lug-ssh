\documentclass{lug}

\title{SSH}
\author{Sumner Evans}
\institute{Mines Linux Users Group}

\usepackage{etoolbox}

\makeatletter
\patchcmd{\beamer@sectionintoc}{\vskip1.5em}{\vskip0.5em}{}{}
\makeatother

\begin{document}

\section{Background}

\begin{frame}{What is SSH?}
    \begin{itemize}[<+->]
        \item SSH stands for \textbf{S}ecure \textbf{SH}ell
        \item SSH is a cryptographic network protocol for operating network
            services securely over an unsecured network.
        \item SSH uses public-key cryptography for authentication
        \item SSH allows you to access your computer remotely and securely
    \end{itemize}
\end{frame}

\begin{frame}{How do I get SSH?}
    \begin{itemize}
        \item Linux: \texttt{ssh} package in your package manager (it's probably
            already installed)
        \item macOS: SSH is already installed, but it may be an old version. Use
            Homebrew to get the latest version.
        \item Windows: You can use PuTTY (\url{http://www.putty.org/})
    \end{itemize}
\end{frame}

\section{Learning Resources}

\begin{frame}{Learning Resources}
    \begin{itemize}[<+->]
        \item test
    \end{itemize}
\end{frame}

\begin{frame}{References}
    \begin{itemize}
        \item Wikipedia: \url{https://en.wikipedia.org/wiki/Secure_Shell}
    \end{itemize}
\end{frame}

\begin{frame}[standout]
    \Huge
    Questions?
\end{frame}

\end{document}
